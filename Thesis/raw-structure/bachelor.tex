\documentclass[11pt,a4paper]{article}

%packages
\usepackage{amsmath}
\usepackage{graphicx}

%opening
\title{Deep Neural Networks for Sound Type Classification}
\author{Xiaowei Jiang}

\begin{document}

\maketitle

\begin{abstract}
	Here is Abstract
\end{abstract}

\tableofcontents


\section{Introduction}
explain structure of this thesis\\
In Chapter 1...\\
In Chapter 2...\\

\section{Background and Motivation}
\subsection{Two-ears System}
brief introduction to two-ears system
\subsection{Sound Type Classification}
\subsubsection{Introduction}
Input features \\
Task\\
\subsubsection{Related works}
Lasso, SVMs classification results\\
Motivation for DNN
\subsection{Convolutional Neural Network Architectures in Sound Type Classification}
explain some basic concepts of CNN
\subsection{Toolbox:Caffe}

\section{Three Types of CNN Architecture in Sound Type Classification(LossFunctions)}
\subsection{One-Against-All}
def.,motivation,result
\subsection{SoftmaxWithLoss}
def.,motivation,result
\subsection{SigmoidCrossEntropy}
def.,motivation,result
introduction to loss function in CNN

\section{Data Augumentation}
\subsection{Overview and Motivation}
definition of data augumentation\\
related works(try different parameters)
\subsection{Implementation}
\subsection{Results}

\section{Dropout Layer}

\section{Optimization Methods}
Discuss about different solver types(SGD,Adam,Adadelta,) related works
\subsection{"SGD"(Stochastic Gradient Descent)}
\subsection{"AdaGrad"(Adaptive Gradient)}
\subsection{"Adam"(Adaptive Moment Estimation)}
\subsection{"AdaDelta"}
\subsection{"NAG"(Nesterov’s accelerated gradient)}
\subsection{"RMSprop"}

\section{Conclusion}

\appendix

\end{document}
