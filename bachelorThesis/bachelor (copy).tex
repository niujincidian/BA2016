\documentclass[11pt,a4paper]{article}

%packages
\usepackage{amsmath}
\usepackage{graphicx}

%opening
\title{Deep Neural Networks for Sound Type Classification}
\author{Xiaowei Jiang}

\begin{document}

\maketitle

\begin{abstract}
	Here is Abstract
\end{abstract}

\tableofcontents


\section{Background and Motivation}
explain structure of this thesis\\
In Chapter 1...\\
In Chapter 2...\\
\subsection{Sound Type Classification}
goal\\
usually features are handcrafted
by domain experts\\
\subsubsection{NIGENS Database}
\subsubsection{Two!Ears System}
idea\\
representations(amsFeatures,ratemaps)\\
{Feature extraction}\\
{Classification}\\
Lasso methods\\
\subsection{Deep Neural Networks}
motivation(DNN can learn features rather than define them->compare to sec.1.2.2)\\
related applications,\\
subsections for DNN(components introduction,caffe)

\subsection{Aims of Thesis}
In my thesis, I investigate different archs and loss functions, data augmentations(overcome overfitting),solver types.\\
also explain motivation\\

\section{Comparison of Architectures and Loss Functions}
\subsection{Methods}
\subsubsection{Architectures and Loss Functions}
Motivations(why use these different Archs->because of the label representation as 0-1 vector)\\
One-Against-All
def.,motivation,result\\
SoftmaxWithLoss
def.,motivation,result\\
SigmoidCrossEntropy
def.,motivation,result,also the experiment with only 'ratemap' features
\subsubsection{Implementation Details}
(Implementations details enable readers to recover the implementation)\\
discuss about parameters
\subsection{Results and Discussion}
Also compare with Lasso results
\section{Data Augmentation}
Overview and Motivation\\
definition of data augmentation\\
related works(try different parameters)
\subsection{Methods}

\subsection{Results and Discussion}

\section{Comparison of Optimization Algorithms}
\subsection{Methods}
Discuss about different solver types(SGD,Adam,Adadelta,) related works
"SGD"(Stochastic Gradient Descent)\\
"AdaGrad"(Adaptive Gradient)\\
"Adam"(Adaptive Moment Estimation)\\
"AdaDelta"\\
"NAG"(Nesterov’s accelerated gradient)\\
"RMSprop"\\
\subsection{Results and Discussion}

\section{Conclusions and Outlook}
summarize the main findings and draw conclusions\\
(drp seems doesn't perform good...)\\
point out future research directions,limitations,
(maybe more drp layers after layers...\\
methods to concour overfitting, more hidden layers,drp,...)

\appendix

\end{document}
